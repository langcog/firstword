\documentclass[10pt, a4paper]{article}
\usepackage[affil-it]{authblk}
\usepackage{multicol}
\usepackage{times}
\usepackage{setspace} 
\usepackage{fullpage}



\begin{document}
\singlespacing

\title{Large scale investigations of variability in early language production}
\author{Rose M. Schneider%
	\thanks{Electronic address: \texttt{rschneid@stanford.edu}}}
\affil{Department of Psychology\\Stanford University}

\author{Dan Yurovsky%
	\thanks{Electronic address: \texttt{yurovsky@stanford.edu}}}
\affil{Department of Psychology\\Stanford University}

\author{Michael C. Frank%
	\thanks{Electronic address: \texttt{mcfrank@stanford.edu}}}
\affil{Department of Psychology\\Stanford University}

\maketitle
\begin{multicols}{2}

%ABSTRACT
\begin{center}
\textbf{Abstract}
\end{center}
The first word, an intimate moment between child and caretaker, exhibits a tremendous amount of variability in semantic categorization, phonological complexity, and age of onset. Through several large datasets of parental report of children’s first words, we investigate patterns in first word production, including the age of onset, distribution of MB-CDI categories, and first words in relation to parental input. In three analyses, we explore the timecourse and distribution of children’s first recognizable language productions. We find that, contra conventional wisdom, more than 75% of children in our datasets produce a first word by their first birthday. In our second analysis, we find that older children produce more words in certain semantic categories. Finally, we take all the unique occurrences of words across the datasets, and try to predict first word production via parental input taken from the CHILDES corpus. Overall, we find that parental report of a child’s first word yields rich and consistent data on what is typically an unobservable dyadic moment, and that the words children produce first may be affected by semantic differences as well as by parental input.

%INTRO
\begin{center}
\textbf{Introduction}
\end{center}
So much intro right here

%GENERAL METHODS
\begin{center} 
\textbf{General Data Collection Methods}
\end{center}
Data for this study is comprised of 4 different datasets, each obtained from a different source. Three of the 4 datasets were drawn from surveys specifically designed for this study. The last dataset contains data from Wordbank, an online repository of data from the MacArthur-Bates Communicative Development Inventories, a widely-used parent-report vocabulary checklist (Fenson et al., 2007). 

%SURVEY 1
\begin{center} 
\textbf{Survey 1: Children's Discovery Museum Survey}
\end{center}

\begin{center} 
\textbf{Participants}
\end{center}
We sent out a brief survey on children’s first words to subscribed members of a large local children’s museum. We received 502 responses to our survey (215 female, 285 male, and 2 with no reported gender; M age = 11 mo, median = 10 mo). Due to the diversity of the San Jose community, several of the first word responses were not in English. Responses were translated into English where possible.

\begin{center} 
\textbf{Methods}
\end{center}
Parents completed a brief web-based survey (created with JavaScript and HTML). The survey asked parents to list their child’s first word (excluding “mama” and “dada”), what they thought word referred to, a description of the situation surrounding the first word, the child’s age at time of utterance (≤10 mo or younger, 11 mo, 12 mo, 13 mo, ≥14 mo), the child’s current age, and their gender. Parents answered for only one child in this survey.

\begin{center} 
\textbf{Data preparation}
\end{center}
Parents’ responses were standardized for ease of analysis. Data cleaning involved fixing obvious spelling errors. When the meaning of the word was not immediately apparent, the researcher relied on the parent’s description of the circumstances surrounding the word and/or the parent’s classification of the word type. Words for which this procedure was not possible were excluded from further analyses (N=XYZ).

%SURVEY2
\begin{center} 
\textbf{Survey 2: Amazon Mechanical Turk}
\end{center}
Our survey in study 2 was an extended version of our previous survey. The survey was programmed in JavaScript and HTML, and posted to Amazon Mechanical Turk externally. This survey allowed for input for multiple children, resulting in our large data set. The Mechanical Turk survey asked parents to input the highest education level of the mother (or primary caretaker), the birth order of the child they wished to answer for, the child’s gender, the first word (excluding “Mama” and “Dada”), the word type, the addressee of the first word, the word age (0 – 24+ months), the current age (0 – 18+ years), the word language, and the home language.  Responses were validated as the survey was completed, reducing the likelihood of erroneous or false responses. 

\begin{center} 
\textbf{Survey 2: Amazon Mechanical Turk}
\end{center}
We recruited 1000 parents from Amazon Mechanical Turk to complete an updated survey on their children’s first words. To minimize the potential for errors in translation of first words, we limited the availability of the survey to Turk parents in the United States. This survey allowed parents to answer for multiple children. We received 1671 responses (813 female, 858 male; M age = 10 mo, median = 10 mo). Approximately 21 children were excluded from subsequent analyses because they had not yet spoken. Responses were translated into English when possible and required.

\begin{center} 
\textbf{Data processing}
\end{center}
Data processing for survey 2 was very similar to survey 1. However, due to our additional questions about language, English translations for the standardized words were much more readily available. Because of the large sample size, many more phonological and morphological variations of first words were given. In these cases, a final standardized form was selected, and the various original first word forms became that standardized form. For example, parents listed “Dog dog”, “Doggy”, “Doggie”, and “Dogie” as the child’s first word; these were all treated as “Dog” in the standardized form. However, as discussed below, there was still some ambiguity in standardizing word forms, and we occasionally had to rely on the parent’s description of the situation of the word occurrence to inform our decisions. The most notable example in this dataset was “Baba” as the original first word response. “Baba” as an original first word had 80 occurrences, but according to the parent’s descriptions very often referred to “Ball,” “Bottle,” or “Grandma.” When the intent of the utterance was clear in the parent’s response, the appropriate standardized form of the word was listed. However, when this was not possible, the standardized form remained “Baba”, resulting in 37 “Baba” standardized first words.

%SURVEY 3
\begin{center} 
\textbf{Survey 3: Contemporary Psycholinguist Diary Studies}
\end{center}

\begin{center} 
\textbf{Methods}
\end{center}
We distributed a short survey via email to a Psycholinguist mailing list. Participants were able to complete this survey more than once for multiple children. Questions included on the survey were: The approximate phonological form of your child’s first word, the age of the utterance, when the parent recorded this (if at all), the child’s sex, the target word, the child’s birth order (first or later born), the child’s current age. 

\begin{center} 
\textbf{Participants}
\end{center}
We received 52 responses from this survey (26 female, 26 male; M age = 11.16 mo, median 11 mo). 

\begin{center} 
\textbf{Data processing}
\end{center}
Data was handled similarly to Surveys 1 and 2. 

%Wordbank
\begin{center} 
\textbf{Methods}
\end{center}

\begin{center} 
\textbf{Participants}
\end{center}

\begin{center} 
\textbf{Info}
\end{center}

%ANALYSES

%Age analysis 
\begin{center} 
\textbf{Age Analyses}
\end{center}
Things go here.

\begin{center} 
\textbf{CDI Categories}
\end{center}
Other things go here

\begin{center} 
\textbf{Input Frequency}
\end{center}

%DISCUSSION

\begin{center} 
\textbf{Discussion}
\end{center}

%ACKNOWLEDGEMENTS

\begin{center} 
\textbf{Acknowledgements}
\end{center}

%REFERENCES

\begin{center} 
\textbf{References}
\end{center}
\end{multicols}
\end{document}
